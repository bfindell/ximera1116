\documentclass{ximera}

% For preamble materials

\usepackage{pgf,tikz}
\usepackage{mathrsfs}
\usetikzlibrary{arrows}
\usepackage{framed}
\usepackage{amsmath}
\pgfplotsset{compat=1.13}

\def\fixnote#1{\begin{framed}{\textcolor{red}{Fix note: #1}}\end{framed}}  % Allows insertion of red notes about needed edits
%\def\fixnote#1{}

\def\detail#1{{\textcolor{blue}{Detail: #1}}}   

\pdfOnly{\renewenvironment{image}[1][]{\begin{center}}{\end{center}}}

\graphicspath{
  {./}
  {ximeraTutorial}
  {chapter1/}
  {chapter2/}
  {chapter3/}
  {chapter4/}
  {chapter5/}
  {proofs/}
  {graphics/}
  {../graphics/}
}

\newenvironment{sectionOutcomes}{}{}


%%% This set of code is all of our user defined commands
\newcommand{\bysame}{\mbox{\rule{3em}{.4pt}}\,}
\newcommand{\N}{\mathbb N}
\newcommand{\C}{\mathbb C}
\newcommand{\W}{\mathbb W}
\newcommand{\Z}{\mathbb Z}
\newcommand{\Q}{\mathbb Q}
\newcommand{\R}{\mathbb R}
\newcommand{\A}{\mathbb A}
\newcommand{\D}{\mathcal D}
\newcommand{\F}{\mathcal F}
\newcommand{\ph}{\varphi}
\newcommand{\ep}{\varepsilon}
\newcommand{\aph}{\alpha}
\newcommand{\QM}{\begin{center}{\huge\textbf{?}}\end{center}}

\renewcommand{\le}{\leqslant}
\renewcommand{\ge}{\geqslant}
\renewcommand{\a}{\wedge}
\renewcommand{\v}{\vee}
\renewcommand{\l}{\ell}
\newcommand{\mat}{\mathsf}
\renewcommand{\vec}{\mathbf}
\renewcommand{\subset}{\subseteq}
\renewcommand{\supset}{\supseteq}
%\renewcommand{\emptyset}{\varnothing}
%\newcommand{\xto}{\xrightarrow}
%\renewcommand{\qedsymbol}{$\blacksquare$}
%\newcommand{\bibname}{References and Further Reading}
%\renewcommand{\bar}{\protect\overline}
%\renewcommand{\hat}{\protect\widehat}
%\renewcommand{\tilde}{\widetilde}
%\newcommand{\tri}{\triangle}
%\newcommand{\minipad}{\vspace{1ex}}
%\newcommand{\leftexp}[2]{{\vphantom{#2}}^{#1}{#2}}

%% More user defined commands
\renewcommand{\epsilon}{\varepsilon}
\renewcommand{\theta}{\vartheta} %% only for kmath
\renewcommand{\l}{\ell}
\renewcommand{\d}{\, d}
\newcommand{\ddx}{\frac{d}{dx}}
\newcommand{\dydx}{\frac{dy}{dx}}


\usepackage{bigstrut}


\title{How is my work scored?}

\begin{document}
\begin{abstract}
  We explain how your work is scored.
\end{abstract}
\maketitle

We want you to learn from this text. Hence, we ask questions to
``push'' your thinking, and leave blanks in examples to ensure you are
following along. We encourage you to \textbf{keep a notebook} where
you write each question and your answers, along with each major
theorem and example. In essence we want you to imagine that \textbf{we
  are writing mathematics together}, and thus are exploring a new
world of mathematics together.

With this in mind, your work is graded on the basis of its correct
\textbf{completion}. The green bar above
\begin{image}
  \includegraphics{partialBar.png}
\end{image}
tells you how close you are to completion. We hope that you can
complete each activity and see a full green bar:
\begin{image}
  \includegraphics{fullGreen.png}
\end{image}
However, sometimes there is a bug that prohibits a ``full green bar.''
In that case, do not despair, as we take these bugs into account when
grading. Moreover, please \textbf{let us know} any issues you are
having. If possible \textbf{we will fix the issue}.

If a correction is made then we may make an update. In this case an orange button will appear at the top of the screen:
\begin{image}
  \includegraphics{update.png}
\end{image}
If you click the ``update'' button, a dialog will appear:
\begin{image}
  \includegraphics{updateDialog.png}
\end{image}
If you update your work, the current activity will be replaced by a
new activity. Since your previous work was for the previous
incarnation of the activity, your previous work will be
deleted. However, if you've completed the activity, \textbf{your
  record of completion remains}. You can witness this by selecting
another activity, and observing your green-bar for the updated
activity.  Unfortunately, if your activity was not complete before you
updated and you want a full green-bar on the updated activity, you
will have to complete the activity again.


We simply want you to learn and to provide you with the best possible
learning experience.

\begin{problem}
  Are you ready to start doing some math!?!
  \begin{multipleChoice}
    \choice[correct]{Yes I am!}
  \end{multipleChoice}
  \begin{hint}
    We promise the real course will not be this corny.
  \end{hint}
\end{problem}


\end{document}
